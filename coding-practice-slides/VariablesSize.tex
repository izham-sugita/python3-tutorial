\begin{frame}[fragile]
\frametitle{Default/Built-in variables type}

\begin{itemize}
\item Python has 5 default/built-in variables:
	\begin{itemize}
	\item character/strings
	\item numbers/numeric
	\item list/tuple/dictionary $\longrightarrow$ actually a derivative
	\end{itemize}

\item The list/tuple/dictionary is a \emph{collection} of character/strings/numbers.
\item Since Python treats \emph{everything} as an object, the derivatives data type can be manipulated like the basic data type too.
\end{itemize}

\end{frame}

%---------------------------------------------------------------------------

\begin{frame}[fragile]
\frametitle{Variables size}
In Python:
\newcommand{\newfilename}{py-number.py}

\lstinputlisting[language=Python, firstline=1, lastline=12]{../py-number.py}

file: \newfilename
\end{frame}

\begin{frame}[fragile]
\frametitle{Variables size}
In C/C++/Fortran:
\newcommand{\newfilename}{datasize.c}

\lstinputlisting[language=C, firstline=1, lastline=14]{../datasize.c}

file: \newfilename
\end{frame}

\begin{frame}[fragile]
\frametitle{Variables size}
\begin{itemize}
\item In Python, int=28, long int=32, float=32, double=32 bytes.
\item Generally a lot more than standard programming language; why?
\item In C/C++/Fortran int=4, long int=8, float=4, double=8 bytes
\item Curious?
\end{itemize}

\end{frame}