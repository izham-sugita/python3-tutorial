% Sections can be created in order to organize your presentation into discrete blocks, all sections and subsections are automatically printed in the table of contents as an overview of the talk
%------------------------------------------------

%\subsection{Subsection Example} % A subsection can be created just before a set of slides with a common theme to further break down your presentation into chunks

\begin{frame}
\begin{center}
\includegraphics[scale=0.25]{python-logo-master-v3-TM.png}
\end{center}
\end{frame}

\begin{frame}
\frametitle{A Little History}
%\lipsum[1]

\begin{itemize}
\item Created by: Guido van Rossum, 1989-1991-ish
\item Why: The creator wanted something easy to use... that's about it.
\item Is it really that easy? Yes (and no)
\item Main resource \url{https://docs.python.org/3/tutorial/index.html}
\end{itemize}

\end{frame}

%------------------------------------------------

%\begin{frame}[fragile] % Need to use the fragile option when verbatim is used in the slide
%\frametitle{The Basic}
%\begin{verbatim}
%import numpy as np
%\end{verbatim}
%\end{frame}

%\begin{frame}[fragile]
%\frametitle{The Basic}
%Module = a file containing Python statements and definitions
%\lstinputlisting[language=Python, firstline=4]{helloPython.py}
%\begin{lstlisting}[language=Python]
%import numpy as np
%\end{lstlisting}
%\end{frame}
%------------------------------------------------
