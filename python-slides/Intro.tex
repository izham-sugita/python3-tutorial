
\begin{frame}
\frametitle{For Your Information}
%\lipsum[1]

\begin{itemize}
\item Created by: Guido van Rossum, 1989-1991
\item Why: The creator wanted something easy to use.
\item Is it really that easy? Yes (and no)
\item Very readable with little memory management.
\item Lots of proven libraries. Open source.
\item \url{https://docs.python.org/3/tutorial/index.html}
\item \url{https://www.tutorialsteacher.com/python}
\item \url{https://www.w3schools.com/python/default.asp}
\item \url{https://www.tutorialspoint.com/python/} $\longleftarrow$ great place to start.
\end{itemize}

\end{frame}

\begin{frame}
\frametitle{A Little More About Python}
%\lipsum[1]

\begin{itemize}
\item Python2 $\longrightarrow$ Python3; support for Python2 will end 2020.
\item Python is fully object oriented. Everything is considered object.
\item Famous for AI and machine learning $\longrightarrow$ Pytorch, Keras, TensorFlow
\item Interpreter language but can be compiled $\longrightarrow$ Cython, Numba
\item Very well documented. Every module/libraries are documented online.
\item Package management by package installer $\longrightarrow$ pip, pip3
\item pip $\longrightarrow$ \url{https://pypi.org/project/pip/}
\item \textbf{Py}thon \textbf{P}ackage \textbf{I}ndex $\longrightarrow$ pypi, \url{https://pypi.org/}
\item \url{https://github.com/} $\longleftarrow$ another place to look.
\end{itemize}

\end{frame}
