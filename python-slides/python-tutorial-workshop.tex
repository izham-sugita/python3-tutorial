%%%%%%%%%%%%%%%%%%%%%%%%%%%%%%%%%%%%%%%%%
% Beamer Presentation
% LaTeX Template
% Version 1.0 (10/11/12)
%
% This template has been downloaded from:
% http://www.LaTeXTemplates.com
%
% License:
% CC BY-NC-SA 3.0 (http://creativecommons.org/licenses/by-nc-sa/3.0/)
%
%%%%%%%%%%%%%%%%%%%%%%%%%%%%%%%%%%%%%%%%%

%----------------------------------------------------------------------------------------
%	PACKAGES AND THEMES
%----------------------------------------------------------------------------------------

\documentclass{beamer}

\mode<presentation> {

% The Beamer class comes with a number of default slide themes
% which change the colors and layouts of slides. Below this is a list
% of all the themes, uncomment each in turn to see what they look like.

%\usetheme{default}
%\usetheme{AnnArbor}
%\usetheme{Antibes}
%\usetheme{Bergen}
%\usetheme{Berkeley}
%\usetheme{Berlin}
%\usetheme{Boadilla}
%\usetheme{CambridgeUS}
%\usetheme{Copenhagen}
%\usetheme{Darmstadt}
\usetheme{Dresden}
%\usetheme{Frankfurt}
%\usetheme{Goettingen}
%\usetheme{Hannover}
%\usetheme{Ilmenau}
%\usetheme{JuanLesPins}
%\usetheme{Luebeck}
%\usetheme{Madrid}
%\usetheme{Malmoe}
%\usetheme{Marburg}
%\usetheme{Montpellier}
%\usetheme{PaloAlto}
%\usetheme{Pittsburgh}
%\usetheme{Rochester}
%\usetheme{Singapore}
%\usetheme{Szeged}
%\usetheme{Warsaw}

% As well as themes, the Beamer class has a number of color themes
% for any slide theme. Uncomment each of these in turn to see how it
% changes the colors of your current slide theme.

%\usecolortheme{albatross}
%\usecolortheme{beaver}
%\usecolortheme{beetle}
%\usecolortheme{crane}
%\usecolortheme{dolphin}
%\usecolortheme{dove}
%\usecolortheme{fly}
%\usecolortheme{lily}
%\usecolortheme{orchid}
%\usecolortheme{rose}
%\usecolortheme{seagull}
%\usecolortheme{seahorse}
%\usecolortheme{whale}
%\usecolortheme{wolverine}

%\setbeamertemplate{footline} % To remove the footer line in all slides uncomment this line
%\setbeamertemplate{footline}[page number] % To replace the footer line in all slides with a simple slide count uncomment this line

%\setbeamertemplate{navigation symbols}{} % To remove the navigation symbols from the bottom of all slides uncomment this line
}

\usepackage{listings}
\usepackage{color}
 
\definecolor{codegreen}{rgb}{0,0.6,0}
\definecolor{codegray}{rgb}{0.5,0.5,0.5}
\definecolor{codepurple}{rgb}{0.58,0,0.82}
\definecolor{backcolour}{rgb}{0.95,0.95,0.92}
 
\lstdefinestyle{mystyle}{
    backgroundcolor=\color{backcolour},   
    commentstyle=\color{codegreen},
    keywordstyle=\color{magenta},
    numberstyle=\tiny\color{codegray},
    stringstyle=\color{codepurple},
    basicstyle=\footnotesize, %\tiny, \small, \footnotesize
    breakatwhitespace=false,         
    breaklines=true,                 
    captionpos=b,                    
    keepspaces=true,                 
    numbers=left,                    
    numbersep=5pt,                  
    showspaces=false,                
    showstringspaces=false,
    showtabs=false,                  
    tabsize=2
}
 
\lstset{style=mystyle}

\usepackage{hyperref}
\usepackage{lipsum}
\usepackage{graphicx} % Allows including images
\usepackage{booktabs} % Allows the use of \toprule, \midrule and \bottomrule in tables

%----------------------------------------------------------------------------------------
%	TITLE PAGE
%----------------------------------------------------------------------------------------

\title[IPSE]{Introduction to Python3 for Scientists and Engineers } % The short title appears at the bottom of every slide, the full title is only on the title page

\author{Muhammad Izham} % Your name
\institute[Universiti Malaysia Perlis] % Your institution as it will appear on the bottom of every slide, may be shorthand to save space
{
Universiti Malaysia Perlis \\ % Your institution for the title page
\medskip
\textit{izham@unimap.edu.my} \\
\textit{sugita5019@gmail.com} \\ %email

%Check out my templates for simple stuffs

\textit{https://github.com/izham-sugita}
}
\date{} % Date, can be changed to a custom date

\begin{document}

\begin{frame}
\titlepage % Print the title page as the first slide
\end{frame}

\begin{frame}
\frametitle{Overview} % Table of contents slide, comment this block out to remove it
\tableofcontents % Throughout your presentation, if you choose to use \section{} and \subsection{} commands, these will automatically be printed on this slide as an overview of your presentation
\end{frame}

%----------------------------------------------------------------------------------------
%	PRESENTATION SLIDES
%----------------------------------------------------------------------------------------

%------------------------------------------------
\section{The Basics} 

% Sections can be created in order to organize your presentation into discrete blocks, all sections and subsections are automatically printed in the table of contents as an overview of the talk
%------------------------------------------------

%\subsection{Subsection Example} % A subsection can be created just before a set of slides with a common theme to further break down your presentation into chunks

\begin{frame}
\frametitle{Introduction}
%\lipsum[1]

\begin{itemize}
\item Created by: Guido van Rossum, 1989-1991
\item Why: The creator wanted something easy to use.
\item Is it really that easy? Yes (and no)
\item Very readable with little memory managment.
\item Lots of proven libraries. Open source.
\item \url{https://docs.python.org/3/tutorial/index.html}
\item \url{https://www.tutorialsteacher.com/python}
\item \url{https://www.w3schools.com/python/default.asp}
\end{itemize}

\end{frame}


\begin{frame}
\frametitle{Installation}

\begin{itemize}
\item Download from \url{https://www.python.org/}
\item For Ubuntu download from repository:\lstinputlisting[language=bash, firstline=1, lastline=1, numbers=none]{./install-commands.txt}

\item For Windows, download from \url{https://www.python.org/downloads/windows/}

\item For Ubuntu installing packages:\lstinputlisting[language=bash, firstline=2, lastline=2, numbers=none]{./install-commands.txt}

\item For Windows installing packages if $C:\backslash Python\backslash Scripts \backslash$ is in the path:\lstinputlisting[language=bash, firstline=3, lastline=3, numbers=none]{./install-commands.txt}

\item For Windows installing packages:\lstinputlisting[language=bash, firstline=4, lastline=4, numbers=none]{./install-commands.txt}

%C:\Python\Scripts

\end{itemize}

\end{frame}

\begin{frame}[fragile]
\frametitle{The Zen of Python}
\begin{itemize}
\tiny
\item Beautiful is better than ugly.
\item Explicit is better than implicit.
\item Simple is better than complex.
\item Complex is better than complicated.
\item Flat is better than nested.
\item Sparse is better than dense.
\item Readability counts.
\item Special cases aren't special enough to break the rules.
\item Although practicality beats purity.
\item Errors should never pass silently.
\item Unless explicitly silenced.
\item In the face of ambiguity, refuse the temptation to guess.
\item There should be one -- and preferably only one -- obvious way to do it.
\item Although that way may not be obvious at first unless you're Dutch.
\item Now is better than never.
\item Although never is often better than *right* now.
\item If the implementation is hard to explain, it's a bad idea.
\item If the implementation is easy to explain, it may be a good idea.
\item Namespaces are one honking great idea -- let's do more of those! 
\end{itemize}

\end{frame}
\begin{frame}[fragile]
\frametitle{The "Hello, World!"}
Greetings!!
\lstinputlisting[language=Python, firstline=4]{../helloPython.py}

file: helloPython.py
\end{frame}
\begin{frame}[fragile]
\frametitle{Variables in Python3}
Importing a module:
\lstinputlisting[language=Python, firstline=1, lastline=1]{../variables.py}

Print to screen:
\lstinputlisting[language=Python, firstline=2, lastline=5]{../variables.py}

Taking an input:
\lstinputlisting[language=Python, firstline=7, lastline=8]{../variables.py}

file: variables.py
\end{frame}

\begin{frame}[fragile]
\frametitle{Variables in Python3}
How to use module:
\lstinputlisting[language=Python, firstline=10]{../variables.py}

file: variables.py
\end{frame}
\begin{frame}[fragile]
\frametitle{Container: List, Dictionaries, Set, Tupples}
\newcommand{\newfilename}{python-container.py}
Container: List
\begin{itemize}
\item A list can contains any type of variable
\item Unlike the normal practice of array where an array contains just one type of variable
\end{itemize}
\lstinputlisting[language=Python, firstline=2, lastline=6]{../python-container.py}

file: \newfilename
\end{frame}
\begin{frame}[fragile]
\frametitle{Class in Python3}
\begin{itemize}
\item Variables in class are public by default.
\end{itemize}
Defining a class:
\lstinputlisting[language=Python, firstline=1, lastline=10]{../py-class0.py}
file: py-class0.py
\end{frame}

%---------------------------------------------------------------------------

\begin{frame}[fragile]
\frametitle{Class in Python3}
\begin{itemize}
\item Variables can be made private.
\end{itemize}
Defining a class:
\lstinputlisting[language=Python, firstline=33, lastline=37]{../py-class0.py}
file: py-class0.py
\end{frame}

%---------------------------------------------------------------------------

\begin{frame}[fragile]
\frametitle{Class in Python3}
\begin{itemize}
\item Initiating class.
\end{itemize}
\lstinputlisting[language=Python, firstline=1, lastline=10]{../py-class1.py}
file: py-class1.py
\end{frame}

%---------------------------------------------------------------------------


%------------------------------------------------
\section{Python3 for Scientists and Engineers}
%------------------------------------------------

\begin{frame}[fragile]
\frametitle{Standard template}
\begin{itemize}
\item This is a standard template slide.
\item Modify by adding items.
\end{itemize}

\end{frame}

\section{Practical Programming}

\begin{frame}[fragile]
\frametitle{Standard template}
\begin{itemize}
\item This is a standard template slide.
\item Modify by adding items.
\end{itemize}

\end{frame}
\begin{frame}
\Huge{\centerline{Thank You!}}
\Huge{\centerline{Questions?}}
\end{frame}

\end{document} 